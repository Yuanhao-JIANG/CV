\documentclass[12pt, a4paper]{article}
\usepackage{hyperref, enumitem, geometry, tabularx, titlesec, multirow, fix-cm}
\geometry{tmargin=0.65in, lmargin=0.6in, rmargin=0.5in, bmargin=0.72in}

\titleformat{\section}{\vspace{-10pt}\scshape\raggedright\large}{}{0em}{}[
    \titlerule \vspace{-6pt}
]

%---------------------------------NEW COMMAND---------------------------------
\renewcommand{\labelitemii}{$\circ$}

\newcommand{\resumeSection}[1]{
    \section*{#1}
}

\newcommand{\resumeSectionItm}[4]{
\item
    \begin{tabular*}{0.96\textwidth}{@{}l@{\extracolsep{\fill}}r@{}}
        \textbf{#1} & #2 \\
        \textit{\textbf{#3}} & \textit{#4}
    \end{tabular*}
}

\newcommand{\resumeEventItm}[2]{
\item
    \begin{tabular*}{0.96\textwidth}{@{}l@{\extracolsep{\fill}}r@{}}
        \textbf{#1} & \textit{#2}
    \end{tabular*}
}

\newcommand{\resumeResearchItm}[3]{
\item
    \begin{tabular*}{0.96\textwidth}{@{}l@{\extracolsep{\fill}}r@{}}
        \multicolumn{2}{@{} l}{\textit{\textbf{#1}}}\\
        \textit{#2} & \textit{#3}\hspace{3mm}
    \end{tabular*}
}

\newcommand{\resumeSectionSubItm}[2]{
\item
    \textbf{\parbox[t]{4.7cm}{#1\hfill}}\parbox[t]{12.5cm}{#2}\vspace{-2mm}
}

\newcommand{\resumeSectionSubItmI}[1]{
\item {#1}\vspace{-2mm}
}

\newcommand{\resumeSectionSubItmII}[1]{\item \textbf{#1}}

\begin{document}
%-----------------------------------HEADING-----------------------------------
\begin{table}[htpb]
    \begin{tabular*}{\textwidth}{@{}l@{\extracolsep{\fill}}r}
        \multirow{3}{25em}{\fontsize{32}{40}\selectfont Yuanhao JIANG}
        & Mobile: 07410581013\\
        & Email: \href{mailto:}{s2132254@ed.ac.uk}\\
        & GitHub: \href{https://github.com/Yuanhao-JIANG}
        {github.com/Yuanhao-JIANG}\\
    \end{tabular*}
\end{table}
\vspace{-4mm}

%------------------------------------INTRO------------------------------------
\resumeSection{Intro}
\begin{itemize}[leftmargin=*]
    \resumeSectionSubItmI{Current undergraduate student with an interest in
        machine learning and statistical modelling, particularly interested in
        generative models and computer vision. Research experience in
        score-based generative modelling with stochastic differential equations
        (SDEs) and reinforcement learning with applications to quantitative
    finance.}
    \resumeSectionSubItmI{Ability to undertake both theoretical and
        computational works, e.g., algorithm and model construction, numerical
        solutions, statistical inference, pure mathematics works including
    algebra, analysis, differential equations, SDEs and so on.}
\end{itemize}
\vspace{-3mm}

%----------------------------------EDUCATION----------------------------------
\resumeSection{Education}
\begin{itemize}[leftmargin=*]
    \resumeSectionItm
    {The University of Edinburgh}{Scotland, UK}
    {BSc (Hons) in Mathematics and Statistics}{Sep. 2020 - Present
    (graduate in May 2024)}
    \begin{itemize}[leftmargin=*]
        \vspace{-1mm}
        \item \it{\textbf{Grades:}}\\
            \it{Year 1: First Class, average grade 91\%}
            \hspace{5mm}\it{Year 2: First Class, average grade 89\%}\\
            \it{Year 3: First Class, average grade 88\%}
            \hspace{5mm}\it{Year 4: Predicted First Class}
        \item \it{\textbf{Prizes and Medals:}}
            \begin{itemize}[
                align=left,
                leftmargin=4em,
                itemindent=0.5pt,
                labelsep=0pt,
                labelwidth=4em
                ]
                \item [\it{2021/22:}]
                    \it{School of Mathematics College Vacation Scholarship}
                \item [\it{2022/23:}]
                    \it{James Ward Prize for distinguished performance in the
                        Degree Examinations in Mathematics \& Statistics}
                \item [\it{2022/23:}]
                    \it{Arthur Erdelyi Prize for distinguished
                    performance in the Degree Examinations for Mathematics}
                \item [\it{2022/23:}]
                    \it{School of Mathematics College Vacation Scholarship}
            \end{itemize}
        \item Final year dissertation focuses on score-based generative
            modelling with stochastic differential equations.
    \end{itemize}
\end{itemize}
\vspace{-5.5mm}
\begin{itemize}[leftmargin=*]
    \resumeSectionItm
    {Hong Kong Baptist University}{Hong Kong}
    {BSc in Mathematics (Year 1)}{Sep. 2019 - May 2020}
    \vspace{-1.5mm}
    \begin{itemize}[leftmargin=*]
        \item \it{\textbf{Grades:}}
            \textit{Year 1 cGPA: 3.72/4}
        \footnotesize \item \textit{\footnotesize Withdrew after successfully
            finishing year 1 and began at the University of Edinburgh}
    \end{itemize}
\end{itemize}
\vspace{-4mm}

%---------------------------------RESEARCH----------------------------------
\resumeSection{Research}
\begin{itemize}[leftmargin=*]
    \resumeResearchItm{Score-Based Diffusion \& Numerical Methods for
    Stochastic Differential Equations}{Supervisor: Prof. Benedict Leimkuhler}
    {May 2023 - Sep. 2023}\vspace{-2mm}
    \begin{itemize}[leftmargin=*]
        \setlength\itemsep{2mm}
        \resumeSectionSubItmI{Applied innovative approaches, especially the
            Leimkuhler-Matthews discretization method, for solving SDEs, to both
            the perturbation process and denoising process. Compared sample
            quality and training efficiency with traditional numerical SDE
            solvers including the Euler-Maruyama method, the Milstein method,
        the stochastic Runge-Kutta method and so on.}
        \resumeSectionSubItmI{Embedded the diffusion coefficient function in
            perturbation SDE with spatial information to allow potentially
        higher perturbation flexibility.}
        \resumeSectionSubItmI{GitHub repository will be released in due course.}
    \end{itemize}\vspace{1mm}
    \resumeResearchItm{
        Mathematics of Reinforcement Learning with Applications to
        Quantitative Finance
    }{Supervisor: Prof. Lukasz Szpruch}{Jun. 2022 - Sep. 2022}\vspace{-2mm}
    \begin{itemize}[leftmargin=*]
        \setlength\itemsep{2mm}
        \resumeSectionSubItmI{Constructed interactive environment to model
        specific quantitative finance scenario.}
        \resumeSectionSubItmI{Implemented various model-free algorithms,
            including Actor-Critic, REINFORCE, and Proximal Policy Optimization
            (PPO), to train our pricing policy. Compared different algorithms
        for training efficiency and effectiveness.}
        \resumeSectionSubItmII{GitHub repository: }
        \href{https://github.com/Yuanhao-JIANG/RL-in-QF}{github.com/Yuanhao-JIANG/RL-in-QF}
    \end{itemize}
\end{itemize}
\vspace{-2mm}

%--------------------------------SKILLS SUMMARY-------------------------------
\resumeSection{Skills}
\begin{itemize}[leftmargin=*]
    \resumeSectionSubItm{Core softwares:}{
        Python, R, Java, Haskell, Git, HTML, CSS, \LaTeX, C, Processing,
        MIPS assembly
    }
    \resumeSectionSubItm{Tools \& Frameworks:}{
        Pacman, Vim, Conda, PyTorch, LWJGL, Bootstrap, Jekyll, ......
    }
    \resumeSectionSubItm{Platforms:}{Linux (Arch Based, Ubuntu), MacOS, Windows}
    \resumeSectionSubItm{Languages:}{English, Chinese (Mandarin)}
\end{itemize}
\vspace{-1mm}


%---------------------------------EXPERIENCE----------------------------------
\resumeSection{Experience}
\begin{itemize}[leftmargin=*]
    \resumeSectionItm{\href{https://www.lms.ac.uk/events/lms-summer-schools}
    {London Mathematical Society Undergraduate Summer School 2023}}
    {}{Summer School Delegate}{16th - 28th Jul. 2023}\vspace{-2mm}
    \begin{itemize}[leftmargin=*]
        \setlength\itemsep{1.5mm}
        \resumeSectionSubItmI{Nominated by the School of Mathematics at the
            University of Edinburgh as one of the 50 students in the UK to
        attend.}
        \resumeSectionSubItmI{Consists of a combination of
            short lecture courses with problem-solving sessions and
            colloquium-style talks from leading mathematicians, covering various
            fields of mathematics including probability theory, statistics,
            information theory, complex analysis, mathematical physics,
            computational number theory, Uncertainty quantification for computer
        models and so on.}
    \end{itemize}\vspace{0mm}
    \resumeSectionItm
    {Careers Service, The University of Edinburgh}{Part time}
    {WeChat Assistant}{Sep. 2021 - Sep. 2022}\vspace{-2mm}
    \begin{itemize}[leftmargin=*]
        \setlength\itemsep{1.5mm}
        \resumeSectionSubItmI{Managed the UoE Careers Service WeChat account
            with more than 1500 students each year, grouped fresh students on
            their matriculation, and maintained alumni groups.}
        \resumeSectionSubItmI{Searched, examined, shared and posted job
        opportunities and career events across UK and China.}
        \resumeSectionSubItmI{Kept in touch with employers, built employers
            groups to provide more and better opportunities with up-to-date
        information.}
    \end{itemize}
\end{itemize}
\vspace{-2mm}

%----------------------------------PROJECTS-----------------------------------
\resumeSection{Personal Projects}
\vspace{1mm}
\begin{itemize}[leftmargin=*]
    \resumeSectionSubItmII{Translation with RNN/Transformer model}
    \vspace{-3mm}
    \begin{itemize}[leftmargin=*]
        \setlength\itemsep{-0.3mm}
        \item A program to train AI translators utilizing RNN or Transformer
            models, able to translate from English to Chinese. The program is
            written in Python with PyTorch and Fairseq (a sequence modelling
            toolkit by Facebook).
        \item Repository: \href{https://github.com/Yuanhao-JIANG/ml-translation}
            {github.com/Yuanhao-JIANG/ml-translation}
    \end{itemize}\vspace{-3.5mm}
    \resumeSectionSubItmII{Handwriting recognition with CNN structure (LeNet)}
    \vspace{-3mm}
    \begin{itemize}[leftmargin=*]
        \setlength\itemsep{-0.3mm}
        \item A handwriting recognition program utilizing a simple convolutional
            neural network, LeNet. The program is written in Python with
            PyTorch package.
        \item Repository:
            \href{https://github.com/Yuanhao-JIANG/ml-handwriting-recognition}
            {github.com/Yuanhao-JIANG/ml-handwriting-recognition}
    \end{itemize}\vspace{-3.5mm}
    \resumeSectionSubItmII{Lightweight game engine}
    \vspace{-3mm}
    \begin{itemize}[leftmargin=*]
        \setlength\itemsep{-0.3mm}
        \item A lightweight Java game engine supports OpenGL. The engine is
            written in Java, and is still under construction.
        \item Repository:
            \href{https://github.com/Yuanhao-JIANG/Java\_game\_engine}
            {github.com/Yuanhao-JIANG/Java\_game\_engine}
    \end{itemize}\vspace{-3.5mm}
    \resumeSectionSubItmII{Light weight parkour game written by Processing}
    \vspace{-3mm}
    \begin{itemize}[leftmargin=*]
        \setlength\itemsep{-0.3mm}
        \item A small parkour game written in Processing. The game is simple
            enough with only about 700 lines of codes, but it is the very first
            program I wrote since I started to learn programming.
        \item Repository: \href{https://github.com/Yuanhao-JIANG/Parkour_game}
            {github.com/Yuanhao-JIANG/Parkour\_game}
    \end{itemize}\vspace{-3.5mm}
    \resumeSectionSubItmII{For more projects visit my GitHub site:
    \href{https://github.com/Yuanhao-JIANG}{github.com/Yuanhao-JIANG}}
\end{itemize}

\end{document}

